\chapter{Budget and impact}
\label{ch:budget}
This chapter covers the economic, social and environmental analysis of the work. The elements which have an associated costs are identified, and they are hardware, software, human resources or general expenses. Also a discussion about the impact of the work from an economic, social and environmental point of view is done.

\section{Budget}
In this section we present the associated costs of the work for each one of the different factors: hardware, software, human resources and general expenses.

\subsection*{Hardware resources}
The hardware resources used to develop this work were a personal computer in the laboratory of perception and manipulation, the WAM arm manipulator, the gripper and the Kinect. The objects used they were simply empty boxes intended to be thrown away therefore they has been not considered in the analysis. 

We consider 20 hours of work per week, and 52 weeks per year.
Regarding the laboratory PC we consider an amortization period of 4 years. The WAM manipulator, due to its high price, is expected to have a higher lifetime and so the amortization period has been chosen to be 10 years. For the gripper we considered an amortization period of 3 years while for the Kinect 4 years.

In Table \ref{tab:hardwareResources} these associated costs of the hardware are shown. These costs are indirect ones.


\begin{table}[h]
  \centering
  \resizebox{\columnwidth}{!}{
    \begin{tabular}{|c|c|c|r|c|r|}
    \hline
    \textbf{Resource} & \textbf{Unit price} & \textbf{Amortization period} & \textbf{Price per hour} & \textbf{Hours of use} & \textbf{Amortization} \\ \hline\hline
    Lab PC                         & €1,000.00           & 4 years                      & €0.12                   & 420                & €50.4                \\ \hline
    WAM                             & €97,500.00          & 10 years                     & €4.84                   & 20 & €96.8 \\ \hline
    Gripper                        & €300.00              & 3 years                       & €0.0481                   & 20                    & €0.96 \\
     \hline 
     kinect & \EUR{150.00} & 4 yeras & \EUR{0.036} & 40 & \EUR{1.44} \\
     \hline \hline
    Total                           & €98,950     & \multicolumn{3}{l|}{-}                                                         & €149.6               \\ \hline
    \end{tabular}
  }
  \caption{Costs associated to hardware resources}
  \label{tab:hardwareResources}
\end{table}


\subsection*{Software resources}
The software used to develop the work were all open source software free of charge. The software to control the robot and to compute the inverse kinematic was developed in the laboratory itself. Therefore for a correct analysis we should consider the hours spent by the worker to design such part of the software. But such a software has been used along many projects done in this laboratory through the years, each one with several hours. Therefore, due to the difficulty to estimate such a cost, we neglected it in the analysis.
\begin{table}[h]
  \centering
  \begin{tabular}{|c|r|c|c|}
  \hline
  \textbf{Resource} & \multicolumn{1}{|c|}{\textbf{Unit price}} \\
  \hline\hline
  Ubuntu 14.04LTS & \EUR{0.00} \\
  \hline
  ROS Indigo (plus packages) & \EUR{0.00}  \\ \hline
 % WAM Interface & \EUR{50.00} \\
  %\hline
    Fast Downward planner & \EUR{0.00}  \\
  \hline
    Point Cloud Library & \EUR{0.00}  \\
  \hline
    Flexible Collision Library & \EUR{0.00}  \\
  \hline
  \LaTeX (\texttt{texlive}) & \EUR{0.00}  \\
  \hline
  Gimp & \EUR{0.00}  \\
  \hline
    \hline
    Total & \EUR{0.0} \\
    \hline
  \end{tabular}
  \caption{Costs associated to software resources}
  \label{tab:softwareResources}
\end{table}

\subsubsection*{Human resources}
\label{sec:hr}
For seek of completeness we 	hypothesize this work could be done by a private institution and we want to analyse the costs of the project by a human resources point of view.  
In the developing of this working there is only one worker, who is the author of this thesis. He performs tasks that are typically 
conducted by individuals with different specialized 
roles. Because of this, the salary for the developer has been adjusted according 
to the amount of time he spent impersonating each role.

\begin{savenotes}%this environment is needed in order to put the footnote at the bottom of the page
\begin{table}[ht]
  \centering
  \begin{tabular}{|c|r|c|r|}
    \hline
    \textbf{Role}      & \multicolumn{1}{l|}{\textbf{Salary (per hour)}} & \multicolumn{1}{l|}{\textbf{Number of hours}} & \multicolumn{1}{l|}{\textbf{Total wage}} \\ \hline\hline
    Technical project manager  & €38.53 \footnote{\href{http://www.indeed.com/salary/Project-Manager.html}{resource \url{http://www.indeed.com/salary/Project-Manager.html}}}  & 50                                           & €1,927   \\ \hline
    Software engineer & €40.6062                                         \footnote{\href{http://www.indeed.com/salary/Software-Engineer.html}{resource \url{http://www.indeed.com/salary/Software-Engineer.html}}}   & 310 & €12,588
    \\ \hline
    
    Tester             & €32.15 \footnote{\href{http://www.indeed.com/salary/Software-Tester.html}{resource \url{http://www.indeed.com/salary/Software-Tester.html}}}   & 60                                           & €1,929 \\ \hline\hline
    Total              & -                                               & 420                                         & €16,444                               \\ \hline
  \end{tabular}

  \caption{Human resources' costs}
  \label{tab:humanResources}
\end{table}
\end{savenotes}

\subsection*{General expenses}
These expenses are relative the use of the laboratory, the energy consumed by the robot and the computer. The consumed energy is a direct and variable cost, while the fixed costs will be the ones regarding the rent of the laboratory. The costs regarding the renting and internet connection have been not considered properly since they are costs of the IRI institute and not of public domain. To include them in the analysis we consider a cost of \EUR{500} per month (the project was developed from March to end of May). To calculate the expenses relative the electricity used we considered the current price in Spain on October 2015 (0.22€/kWh)\footnote{resource \href{http://www.elmundo.es/economia/2015/10/20/5626187fca474195608b45c7.html}{\url{http://www.elmundo.es/economia/2015/10/20/5626187fca474195608b45c7.html}}}.

\begin{table}[h]
  \centering
  \begin{tabular}{|c|c|c|r|}
    \hline
    \textbf{Resource} & \textbf{Average power} & \textbf{Hours of usage} & \textbf{Price} \\ \hline\hline
    Lab PC            & 250 W                  & 420 h                   & \EUR{23.10} \\ \hline
    WAM               & 60 W                   & 20 h                    & \EUR{0.26}          \\ \hline\hline
    \textbf{Total}    & \multicolumn{2}{l|}{}                            & \EUR{23.36}        \\ \hline
  \end{tabular}
  \caption{Electricity consumption and energy cost. A price of €0.22 per kWh has been assumed.}
  \label{tab:electricityConsumption}
\end{table}

\FloatBarrier

\subsubsection*{Total cost}
Considering the costs of the hardware, software, human resources and general expenses we obtain the total cost of the project.  We consider also a margin of contingencies of 15\% in order to consider additional costs and the error of the estimations of the costs. 
The estimated costs, when no data were available were overestimated. 

\begin{table}[h]
  \centering
  \begin{tabular}{|c|c|}
    \hline
    \textbf{Category} & \textbf{Value} \\ \hline\hline
    Hardware    & € 149.6            \\ \hline
        Software    & € 0             \\ \hline
        HR    & € 16,444             \\ \hline
        Variable costs    & €23.36  \\ \hline
        Fixed costs    & €500.0  \\ \hline  
        \hline  
                Subtotal costs    & €17,116.96  \\ \hline  
                Margin (15\%)    & €2,567.54  \\ \hline  
Total costs & €19,684.504  \\ \hline  
  \end{tabular}
  \caption{Total costs}
  \label{tab:internetCost}
\end{table}



\section{Impact}
\subsection*{Economic impact}
The work developed in this thesis is not a product to sold and have no direct benefits. This thesis had the aim to investigate a new method in order to contribute with the current state of the art in manipulation planning. The impact of this work is increasing the amount of knowledge and contributions created in the UPC. Regarding the developing of the project inside the UPC the project should not taken into account the human resources of Section \ref{sec:hr} since the project is performed by a student with no grants, therefore the cost of the project is about \EUR{672.96} (overestimated). The budget is therefore low and we think that the project is economic sustainable.

The application of a robot to solve manipulation problem, such grasping a specific object in a cluttered environment, in industrial set ups is still far to be applied. The robots, although expensive, can substitute an human operator saving money but the time they take to solve manipulation tasks is much more than one human would take. This consideration is based on the experiment we performed, the task could be easily solved by a human in few seconds while the robots took about 5 minutes. Although the planning system showed to work fine the problems regarding the execution time make this approach not even considerable for a real application in industries.

\subsection*{Environmental impact}
Considering the directive 2011/92/EU of the European Parliament and of the council of 13 December 2011\footnote{\href{http://eur-lex.europa.eu/LexUriServ/LexUriServ.do?uri=OJ:L:2012:026:0001:0021:En:PDF}{\url{http://eur-lex.europa.eu/LexUriServ/LexUriServ.do?uri=OJ:L:2012:026:0001:0021:En:PDF}}}
they sates in article 1 point 1 the following:
\begin{quote}
The directive shall apply to the assessment of the environmental effects of those public and private projects which are likely to have significant effects on the environment.
\end{quote}
Since the project presented in this thesis is a project of research with no direct applications in industry, it has no significant effects on the environment and an environmental impact analysis is not required.

\subsection*{Social impact}
The result of this work has no direct impact on the society in the short time horizon. As commented, this is a research project which wants to enrich the current state of the art in manipulation planning. Hopefully, this work will inspire some members of scientific community leading them, in a long time horizon, to develop an efficient planning system to solve a wide array of manipulation problems. This could be increase the use of robots in industrial environments or in house environments, a branch of the scientific community is investigating with the aim to bring the robots in houses to assist humans. From this point of view, this work could be have a social impact on the society in a long time horizon, but it is difficult to predict since it depends on how it will be useful to develop an effective technology. 
