\chapter{Task Planner}
\label{ch:task_planner}
In this chapter the general framework adopted is discussed, proposing a suitable task planner. After the  review of the current state of the art of task planners, a proper planner is chosen and then a suitable description to the table clearing problem is discussed.
\section{State of the Art}
As already seen in chapter \ref{ch:state_of_the_art} there exist different kind of planners and they can be grouped in three main categories:
\begin{enumerate}
\item classical planners
\item hierarchical planners
\item probabilistic planners
\end{enumerate}
\textbf{Classical planners}, as suggested by the name, are the more classical and easy to use. They are characterized by environments which are fully observable, deterministic, finite and static (changes happen only when the agent acts) and discrete (in time, actions, objects..) \citep{artificialIntelligence}.
A deterministic problem is generally formulated as a $6$-tuple $\Pi=\langle S^d, s_o^d, G^d, A^d, T^d, c^d \rangle$ \cite{little2007probabilistic}, where:
\begin{itemize}
\item $S^d$ is a finite set of states;
\item $s_o^d \in S^d$ is an initial state;
\item $G^d \in S^d$ is a goal state;
\item $A^d(s)$ is a set of applicable actions for each $s \in S^d$;
\item $T^d(a,s) \in S^d$ is a deterministic transition function for all actions $a \in A^d(s)$ and states $ s \in S^d$;
\item $c^d(a)$ is the cost to apply action $a$. 
\end{itemize}
The solutions, or trajectorys, $\tau_i$ are sequences of actions applicable from the initial state until the goal state. The cost of a trajectory $C(\tau_i)$ is the sum of the cost of the actions of the trajectory $\sum_{a \in \tau} c^d(a)$. The optimal solution is the solution with less cost: $\tau^* = \min_{\tau_i} c^d(\tau_i)$. A very well known classic planner is the Fast Downward planner \cite{helmert2006fast}.

\textbf{Hierarchical planning}, also called \textit{Hierarchical Task Network}(HTN), works in a similar way to how it is believed that human planning 
works \citep{marthi2007angelic}. It is based on a reduction of the problem. The planner recursively decomposes tasks into subtasks, stopping when it reaches primitive tasks that can be performed directly by planning operators. In order to tell the planner how to decompose nonprimitive tasks into subtasks, it needs to have a set of methods, where each method is a schema for decomposing a particular kind of task into a set of subtasks \citep{shopDescription}. For this kind of planning technique a well known planner is JSHOP2 \cite{JSHOP2}.
\textbf{Probabilistic planning} is a planning technique which consider that the environment is not deterministic but probabilistic. So the actions have a certain probability to obtain a certain state, and given an initial and final state the planner  finds the solution path with the highest probability. A well known example on which this kind of planners are build on is the Markov Decision Process. A probabilistic problem is generally formulated as a $6$-tuple $\Pi=\langle S, s_o, G, O, T, A \rangle$ \cite{little2007probabilistic}, where:
\begin{itemize}
\item $S$ is a finite set of states;
\item $s_o \in S$ is the initial state;
\item $G \in S$ is a goal state;
\item $O$ is the set of outcomes, the probability of $o \in O$ is $Pr(o)$;
\item $T(o,s) \in S$ is a (total) deterministic transition function for all outcomes $o \in O$ and states $s\in S$;
\item $A(s)$ is a set of applicable actions for each $s \in S$, coupled to a function $out(a) \subseteq O$ mapping each action to a set of outcomes in such a way that 
\begin{itemize}
\item each outcome $o \in O$ belongs exactly to one action $act(o)$;
\item $\sum_{o \in out(a)} Pr(o) = 1$ for all $a$.
\end{itemize}
\end{itemize}
In this case the optimal solution is the one with the highest probability. In this category two famous probabilistic planners are Gourmand \cite{Gourmand} and PROST \cite{PROST}.

\section{Planner}
The problem this thesis is facing could be resolved by several approaches by using planners from all the categories. We have already seen that such a problem involves geometric constraints and those cannot be considered directly by the planner using a ready to use state of the art planner, that would imply a designing of a new planner. Since the aim of this work is not to design a new planner but to resolve the table clearing task through already existing planners the problem has to be cast in a way to be manipulated by existing planners. This easy way involves working with symbolic predicates, symbolic predicates are predicates which can be true or false, and they will be introduced more in detail in the next sections. 

The problem moreover involves a big amount of uncertainty due to the interaction of the robot with the environment. When the robot will interact with the pile of objects it is very hard to predict correctly the position of the object in the next frame, that is the next state, this is a crucial problem which should be considered. With a probabilistic planner, the planner will take into account what object has been moved and it will update the state obtaining a set of states, each one with a certain probability, and the returned plan, or solution, is the one with the highest probability. The probability also has to be modelled and it is highly depending on the form of the object, which is also hard to predict. An other way to face the problem is to replan at each frame, that is after each interaction of the robot with the pile of objects, or whenever the current state deviates from
the expected one, generating a new trajectory from
the current state to the goal. The plan's actions are considered deterministic, and the only useful action is actually the first one of the plan, after its execution the system replans again. Little et al. discussed in \cite{little2007probabilistic} the problem of when is more useful the probabilistic planning with respect a simple replanning system. 
They defined a the concept of \textit{Probabilistic Interesting Problem} with the following definition:

\begin{displayquote}
A probabilistic planning problem is considered to be \textit{probabilistically interesting} if and only if it has all of the following structural properties:
\begin{itemize}
\item there are multiple goal trajectories;
\item there is at least one pair of distinct goal trajectories, $\tau$
and $\tau'$, that share a common sequence of outcomes for the
first $n-1$ outcomes, and where $\tau_n$ and $\tau_n'$ are distinct
outcomes of the same action; and
\item there exist two distinct goal trajectories $\tau$ and $\tau'$ and outcomes $o \in \tau$ and $o' \in \tau'$ of two distinct actions $a = act(o)$ and $a' = act(o')$  such that executing a strictly decreases
the maximum probability of reaching a state where a can
be executed.
\end{itemize}
\end{displayquote}
They assert that unless a probabilistic planning problem satisfies all of the structural conditions in this definition, then
it is inevitable that a well-written replanner will outperform
a well-written probabilistic planner. Moreover the authors do not negate the possibility that a deterministic replanner could perform optimally even for probabilistically interesting planning problems. 

To conclude, a replanner would make more probabilistic problem practically solvable. In the other hand it suffers to be less robust than a probabilistic one. 


\begin{comment}
From their analysis the state that probabilist planning has some useful properties which a deterministic replanning does not have, and they are:
\begin{itemize}
\item The lack off \textit{dead end} states, from which the goal is unreachable through any
combination of chance and choice,
\item  the degree to which the probability of reaching a \textit{dead end} state can be reduced through the choice of actions,
\item a higher degree in the solution trajectory, yielding to possible several and feasible solutions,
\item  the presence of mutual exclusion
\end{itemize}
\end{comment}

Taking into account such considerations, in order to face simply such a difficult problem, the problem has been thought to be solved thanks a deterministic replanner although it is \textit{probabilistic interesting}. The choice also was guided by the difficulty to model the probability distribution of the actions which depends on the particular shape of the objects the robot has to interact with. 

Between a classic planner and an heuristic one there is not much difference for the aim of this work.
The planner chosen to develop the work is the \textbf{Fast Downward} planner \cite{helmert2006fast}\cite{FastDownwardPage}, a very well know classic one. The advantage of this planner is its wide documentation with respect other planners, which makes it easier to use and it also has a wide community. 


\section{PDDL}