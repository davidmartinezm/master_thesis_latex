\chapter{Conclusions}
\label{ch:conclusions}

\section{Conclusions}

\section{Limitations}
The planner here presented has some clear limitations, these are related to the lack of the probability and mainly to the lack of geometric information. 
How previously said, the pushing action is supposed to push the object far enough from the others ones, this will be true only if the pushing will be performed up to infinite, and this is not the case.. The pushing is performed taking into account the geometry of the manipulated object, without taking care about the surrounding objects geometry and poses. This problem is though to be resolved by replanning, that is, if an object has been pushed but not enough, the planner will return a sequence where the first action is again to push that object. This is actually what the planner does, but since for the most of cases, pushing along directions $1$ and $2$ is feasible, for the planner both directions are feasible to resolve the problem. So what it could happen is that the planner could first return as solution to push a certain object along direction 1, and then, when replanning, to push the object along direction 2. What is here commented is a infinite loop. This happens because the planner has no information between consecutive frames, that is, it consider the problem as a separated one from the previous frame. 
This fact is a very undesirable one and there are several scenarios that can present a behaviour like that one. The promising part is the combination of the grasping and pushing actions. It has been observed that the majority of times, although for complex scenarios, the solution is mainly based on a proper sequence of grasping action, while the pushing action is an auxiliary action which is rarely used. Taking this into account, and the presented limitation, is very likely that once an object has been pushed, ones of the objects which cannot be grasped before can now be grasped, avoiding such undesired infinite loop.  


\textbf{Somewhere talk about the limitation of the segmentation that is we assume it to be perfect because if not the planner will return likely an infeasible plan}

\section{Future Work}