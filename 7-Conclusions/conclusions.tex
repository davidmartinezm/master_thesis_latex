\chapter{Conclusions}
\label{ch:conclusions}

In this chapter we present the conclusions of this work, and also the limitations and possible lines of future work.

The objectives established have been achieved developing a planning system which is able to solve table clearing tasks with cluttered objects reasoning at a symbolic level. The planning system can choose the best sequences of pushing and grasping actions to solve problems that could be not solved by only considering grasping actions.

The novelty of this thesis regards the tackling of geometrical restrictions within symbolic planning. To do so, a perception system, which understands what are the objects that hinder some actions, has been proposed. Such a perception system allows the planning to reason at a semantic level. Other geometrical constraints such as the inverse kinematic are efficiently considered through backtracking speeding up the reasoning process. To deal with unexpected outcomes replanning is used. As shown in Figure \ref{fig:pipeline} the whole process is quite fast in deciding the next action to do. 

This work showed as some complex manipulation problems, such the one presented, can be solved by a symbolic planner, handling some geometrical constraints with symbolic predicates and others through backtracking.

%\todo[inline]{Can i say novelty in this context?}
%\DMcomment{It seems that you are saying that the main contribution was the perception. You can say novel, but the novel part is the way you tackle geometrical restrictions within planning, for which you have to use the perception that you have proposed. I think you should rewrite and work more on this paragraph as it is very important.}
%The novelty is regarding a high level perception algorithm which generates the states required to understand how the objects need to be manipulated. Then a suitable domain description is proposed to solve the task. The state\DMC{s} generation and the planning showed to be efficient in solving the task correctly and fast. 
%The geometric constraints due to the inverse kinematic are efficiently solved by the backtracking technique, speeding up the reasoning process. Considering Figure \ref{fig:pipeline}, the planning system is quite efficient although its fastness depends on several other elements of the pipeline such as the inverse kinematic and the low level perception (filtering and segmentation).  

The planning system has been inspired by the way humans solve the task and the experiments showed the robot can solve the task with a intelligent sequence of actions that we consider close to the one a human would do. 

The planning system can be easily adapted to every kind of robotic manipulator and gripper, as it only needs to include the model of the gripper and the inverse kinematic. It can also be easily integrated with more sophisticated grasping techniques.

%This work also showed as a complex manipulation problem, such the one presented, can be solved by a symbolic planner, considering geometrical constraints with backtracking. \DMcomment{You should say the contribution in the main contributions paragraph before.}

 



\section*{Limitations}
Although the system showed to work good it has some limitations which could make it no adapt, at least without further improvements, for a real set up. 

A limitation is related to the lack of geometrical information during the planning stage. 
As previously said, the pushing action is supposed to push the object far enough from the others ones, this will be true only if the pushing will be performed up to infinite, and this is not the case. The pushing could move the object not enough in order to make it graspable, this is though to be solved by replanning and the planners will returns to push that object again.
In that situation it could happen the planner pushes first in one direction and then in the opposite one, in this way it gets stuck in a loop. However, during our experiments, this situation was really rare. 

Moreover, the planning system does not take into account the objects when the manipulator arm moves. This could likely lead to undesired collisions and the integration of a obstacle avoidance path planner should be considered in the action execution stage. 

The biggest limitation is that the planning system relies on a good segmentation. A bad segmentation could not lead to any feasible plan and the parameters should be tuned for each particular scene in order to achieve a fairly good segmentation. The planning system considers that the segmentation is perfect and it does not know how to deal in cases the segmentation is bad. This limitation reduces the scope of the system since without a proper segmentation the system is no more effective. 

The scope of this system is limited to scenarios with objects which have simple shapes (cylinders or parallelepipeds), this is mainly due to the limitation of the decision regarding the direction along with push an object.  


\section*{Future Work}
As future work we would like to include some geometrical information in the pushing action, that is knowing how much the robot should push the objects. To do so, the \ttt{block\_dir} predicate could be computed by translating the object along the pushing direction up to finding a pose for that object in which the grasping pose is not colliding with any object. In this way the robot will push the object away with the aim to put it in a pose that it can be grasped.

The integration of \textit{MoveIt!}\footnote{Ioan A. Sucan and Sachin Chitta, “MoveIt!”, [Online] Available:\href{http://moveit.ros.org} {\url{http://moveit.ros.org}}} will be also considered in the action execution stage in order to avoid collisions. 

The robot may execute actions that avoid collisions by just few millimetres, but the noise of the Kinect and the controllers might make the robot collide. Therefore a cost could be added to the actions when planning to prefer safer actions with wider collision-free ranges.