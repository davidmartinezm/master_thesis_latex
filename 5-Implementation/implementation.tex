\chapter{Software design}
\label{ch:software_design}
In this chapter the design of the software is briefly described and the  software used as well. 

The code has been implemented in ROS framework (Robot Operating System) \citep{ROS} using \ttt{C++} as programming language. 
External libraries used are Point Cloud Library (PCL) \citep{PCL}, an open source library which provides a wide variety of tools for 3D perception, and the Flexible Collision Library (FCL) \citep{pan2012fcl}, an open source  library for collision checking.

The algorithm is mainly based on the PCL library which is used to do the following operations:
\begin{itemize}
\item Filtering
\item Segmentation
\item Plane estimation
\item Principal Component Analysis
\item Projections onto the table plane
\item Convex Hulls
\end{itemize}
The FCL library was used only for collision detection between the convex hulls of the objects. 

The planner used is the Fast Downward planner \citep{helmert2006fast}. To get a plan the binary file of planner is called giving as inputs a domain and a problem description in PDDL syntax.

 
\iffalse
For the collision checking the Flexible Collision Library (FCL) \citep{pan2012fcl} has been used. This library allows to define the collision problem in a simpler manner than other more famous collision libraries such as \textit{Bullet} \citep{Bullet}, and it can work with different objects shapes such as box, spheres, cone, convex, mesh and octree. 
The main library used in this work is the Point Cloud Library (PCL) \citep{PCL_}, which allows some methods to create an object shape from a point cloud.

\begin{enumerate}
\item Nodes
\item Ros graph 
\item simulation
\item PCL - FCL 
\end{enumerate}
\fi

\paragraph{ROS}
The algorithm has been developed by using different nodes in order to have a modular code. 
The nodes implemented are:
\begin{itemize}
\item A node to segment the objects and estimating the table plane coefficients,
\item a node to generate the states having as input the segmented objects and the table plane,
\item a node  to write the problem in PDDL syntax, given the states, and to call the Fast Downward binary file and reads the returned plan, 
\item a node to execute the pushing or grasping action,
\item All those nodes are connected each other by a decision maker node which is the one which controls all the process.  
\end{itemize}
The decision maker nodes is the one which receives the point cloud, call all the other services, decides what action to execute accordingly to the plan and so on. 

\paragraph{Simulation}
Before to test the code implementation with the real robot it was first tested in simulation with Gazebo \citep{koenig2004design}. Unfortunately, since the gripper was not modelled in Gazebo only the pushing action was tested in simulation. For the grasping action a simple urdf model of the gripper was designed and checked only if the robot performs correctly the movements for the grasping action.  